\chapter{Considerações Finais} \label{cap:cap7}

O uso de um simulador no treinamento ajuda a atenuar riscos de falhas da raquianestesia relacionados a habilidades médicas não adquiridas corretamente assim como reduzir custos de laboratório e peças pra reposição de modelos físicos. Foram simulados aqui os comportamentos necessários para representar a agulha espinhal penetrando através dos vários ligamentos e tecidos do corpo. Um simulador de raquianestesia de alta fidelidade requer recursos como haver uma região das costas com a sensação de vértebras da coluna que possam ser localizadas por apalpação, a capacidade de acomodar várias posições do paciente, inclinações ajustáveis de inserção da agulha e alguns outros aspectos principais. Todos estes itens citados em específico são contemplados no simulador criado aqui que inclui um modelo 3D criado e utilizado juntamente com os parâmetros e configurações descritas nesta tese. 

Nos experimentos, o objetivo foi de simular corretamente as sensações táteis essenciais da punção da agulha durante o procedimento de raquianestesia. A identificação das transições entre os tecidos foi um dos comportamentos levado em consideração, por ser imprescindível na representação do caso real e, portanto, no treinamento deste procedimento médico.
A variação da resistência ao movimento da agulha em cada tecido foi outro comportamento considerado. O resultado dos experimentos mostrou que as opções de configurações presentes nas ferramentas táteis programáveis são capazes de representar esses comportamentos.

Não foram feitas avaliações do uso do ambiente de simulação desenvolvido nesta tese em relação a ergonomia, porém, foram feitas adaptações da montagem dos equipamentos nos testes com especialistas. Estas variações na montagem foram feitas por sugestão dos próprios anestesistas para que eles executassem o procedimento no ambiente de treinamento em posição similar aquela que estão acostumados a conduzir o procedimento real.

\section{Conclusão}
\label{sec:conclusão}

Nesta tese de doutorado foi criado ambiente virtual para treinamento do procedimento de raquianestesia em gestantes desde a apalpação até a administração do anestésico. A solução proposta para a Questão~\ref{prob:simuladorCasoReal} apresenta um ambiente com avaliação do desempenho dos usuários de acordo com suas sequências de execução dos procedimentos bem como apresenta uma forma totalmente virtual para representação da palpação da coluna através do uso de um dispositivo háptico.

Dentre as contribuições deste trabalho está a reprodução virtual das principais sensações hápticas necessárias para simular a anestesia raquidiana. Foram feitos experimentos para avaliar a eficiência da simulação a partir do uso das ferramentas hápticas disponíveis. O sistema foi avaliado quanto a sua usabilidade de forma quantitativa através do método \textit{\acrshort{SUS}} \cite{Brooke2013} tendo como resposta a classificação de usabilidade boa. Especialistas fizeram uso do ambiente de simulação e responderam questionários de avaliação. Estes informaram considerar como útil este tipo de treinamento. Os anestesistas recomendaram algumas melhorias visando aproximar o simulador ainda mais do procedimento real e desta forma poder ser de fato utilizado como ferramenta de treinamento de anestesia raquidiana. Na comparação com o simulador de Färber et al. \cite{Farber2008} as avaliações do simulador desta tese obtiveram notas inferiores, porém, como comentado na Seção~\ref{sec:avaliacaoEspecialistas}, não foi possível reproduzir o mesmo cenário e nem o mesmo equipamento (dispositivo háptico) nos testes. É importante ressaltar que, apesar das notas menores, o simulador desta tese tem duas das características principais de simuladores de punção mais completos que o simulador de Färber et al. \cite{Farber2008} não possui. O simulador apresentado nesta tese possibilita a palpação da coluna para escolha do ponto de inserção da agulha, esta característica só existe em outros dois dos simuladores computacionais estudados nesta tese (nenhum destes simula a anestesia raquidiana). A outra característica é que com o modelo 3D adaptável desenvolvido nesta tese é possível a simulação de múltiplas pacientes. O simulador de Färber et al. \cite{Farber2008} possui somente três (3) representações de pacientes.

Outra contribuição que é uma etapa fundamental para tornar a simulação mais real e adaptável a vários cenários de testes foi a construção de um modelo 3D do tronco de uma gestante de forma facilmente modificável via programação. A variabilidade de pacientes que pode ser configurada usando o modelo proposto é bastante vasta. Para uma maior reprodução do caso real no que diz respeito à espessura das principais camadas o modelo foi alimentado por um equação genérica desenvolvida neste trabalho. Foi também descrito o passo a passo das formas de uso da \textit{engine} da \textit{Unity} e do \textit{plugin} para importação do modelo e interação deste com o háptico na simulação. 

Em relação à evolução dos simuladores atualmente disponíveis para raquianestesia a contribuição desta tese envolveu a simulação virtual da palpação. Outra contribuição neste sentido foi a construção de um ambiente para treinamento de raquianestesia (e não somente um simulador). Este ambiente apresenta \textit{feedbacks} durante e após cada procedimento. Propomos aqui uma avaliação do desempenho de cada procedimento por meio de notas bem como a avaliação combinada de todos os procedimentos feitos pelo usuário. 

\section{Limitações}

Descrevemos nesta Seção as principais limitações deste trabalho. A primeira se refere ao dispositivo háptico que tivemos a oportunidade de adquirir e utilizar no desenvolvimento deste trabalho. O dispositivo é um modelo mediano, não exatamente com todos os graus de liberdade e a capacidade de reprodução das forças necessárias a uma completa representação do que seria uma movimentação da agulha. O dispositivo utilizado nesta tese não permite a simulação de forma completa de todas as restrições de movimento para representação da movimentação da agulha após a mesma ser inserida no paciente. Devido ao alto custo, não foi possível viabilizar o uso de um modelo mais avançado que possui mais graus de liberdade e possibilidade de representação de força, rotações e torções mais adequadas. Isto fez com que parte da realidade do procedimento ficasse comprometida. Em relação aos graus de liberdade de movimento, esta limitação impactou o realismo, pois quando uma agulha é inserida na pele (no procedimento real) existe uma força que impede o movimento de rotação lateral da agulha que não é possível de ser representado pelo equipamento utilizado nesta tese. Com relação à força máxima suportada pelo aparelho, o impacto no realismo aconteceu quando voluntários, ao usarem o simulador, sentiam uma resistência ao avanço da agulha ou ao toque na pele. Neste caso a reação do voluntário em alguns casos foi a de aumentar a força aplicada, ultrapassando o limite suportado pelo háptico. Quando esse tipo de situação ocorre, o equipamento é projetado para liberar o avanço de forma a evitar danos ao seus mecanismos (procedimento de segurança do dispositivo) o que ocasiona uma perda da imersão da sensação háptica.

Existem outras limitações relacionadas à forma como foram realizados as pesquisas que obtiveram as respostas aos questionários na última fase dos testes realizadas no Hospital Universitário Antônio Pedro (HUAP). Uma delas foi a restrição de tempo dos especialistas para realizarem os testes. Outra foi a ausência de um grande número de voluntários residentes ou seja estudantes ainda em treinamento em anestesias, o que inviabilizou uma reprodução de experimento mais justa no que diz respeito a comparação que foi efetuada com o similador desenvolvido na pesquisa de Färber et al. \cite{Farber2008}. Para viabilizar o acesso ao maior número de especialistas para avaliação do ambiente de simulação foi necessário efetuar a montagem do háptico e do laptop na secretaria do centro cirúrgico do HUAP, aproveitando o tempo dos especialistas no intervalo entre cirurgias. O cenário ideal seria uma montagem num local reservado sem assistência, visão, ou influência de colegas no momento da simulação, como acabou acontecendo no teste de parte dos voluntários. 

\section{Trabalhos Futuros}
\label{sec:trabFuturo}

Uma próxima etapa idealmente envolveria profissionais instrutores de procedimentos de raquianestesia de forma a validar a possibilidade do uso do simulador no treinamento. Após um passo inicial de avaliação para inclusão de novas funcionalidades ou remodelagem de funcionalidades existentes adequando este às necessidades de treinamento. Alguns exemplos de adequações a serem feitas mesmo sem esse contato com especialistas poderiam envolver por exemplo a representação virtual do escorrimento do líquor quando a agulha fica por alguns segundos no espaço subaracnóideo e a modelagem de toda a paciente incluindo as demais partes do corpo para um ambiente ainda mais completo em termos de imersão. Ainda visando uma representação mais fiel da realidade a inclusão de \textit{feedback} de dor da paciente para possíveis movimentos bruscos de objetos perfurantes no seu corpo pode ser incluída, uma opção é usar um sintetizador de voz para este fim. A inclusão de óculos 3D (uma das principais questões trazidas na avaliação pelos especialistas) traz benefícios em relação à imersão ainda que tendo o custo deste equipamento. O uso deste tipo de equipamento é uma realidade possível em hospitais e universidades com melhor estrutura para treinamento.
Em relação à generalização e aproveitamento do trabalho feito na construção do modelo 3D é possível fazer o uso deste modelo para outros tipos de procedimentos que forem efetuados na mesma área do corpo (dorso). Este foi mais um ponto levantado por diversos especialistas falando sobre outros tipos de procedimentos onde este tipo de simulador também faria sentido.

A aplicação de técnicas para reconhecimento de engajamento por parte do usuário \cite{Mitsis2022} é uma opção que poderia ser usada para alterar rumos do treinamento.
Outro caminho de trabalho futuro está no uso de realidade aumentada no lugar do dispositivo háptico usando a identificação da mão e incluindo esta no ambiente virtual. Nesta abordagem no momento da punção seria incluído a agulha ou seringa na mão da pessoa usando o simulador. Para tornar possível esta prática é preciso o estudo de formas de ``enganar'' o cérebro humano para que este perceba a resistência ao avançar da mão no ambiente virtual mesmo sem que algo físico impeça este movimento (como acontece no caso real e com a simulação com o háptico). Uma vez encontrada esta solução a abordagem ganharia em flexibilidade e teria o custo reduzido por conta da ausência da necessidade de existência do dispositivo háptico. Um rumo que envolve tecnologias recentes para esta área seria o de modelar uma nova arquitetura do ambiente de treinamento com a integração de conceitos de internet das coisas \cite{Ahmad2022}.