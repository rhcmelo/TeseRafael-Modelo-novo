\chapter{Considerações Finais} \label{cap:cap6}

O uso de um simulador no treinamento ajuda a atenuar riscos de falhas da raquianestesia relacionados a habilidades médicas não adquiridas corretamente assim como reduzir custos de laboratório e peças pra reposição de modelos físicos. Foram simulados aqui os comportamentos necessários para representar a agulha espinhal penetrando através dos vários ligamentos e tecidos do corpo. Um simulador de raquianestesia de alta fidelidade requer recursos como haver uma região das costas com a sensação de vértebras da coluna que possam ser localizadas por apalpação, a capacidade de acomodar várias posições do paciente, inclinações ajustáveis de inserção da agulha e alguns outros aspectos principais. Todos estes itens citados em específico são contemplados no simulador criado aqui que inclui um modelo 3D criado e utilizado juntamente com os parâmetros e configurações descritas nesta tese. 

Nos experimentos o objetivo foi de simular corretamente as sensações táteis essenciais da punção da agulha durante o procedimento de raquianestesia. A identificação das transições entre os tecidos foi um dos comportamentos levado em consideração por ser imprescindível na representação do caso real e, portanto, no treinamento deste procedimento médico.
A variação da resistência ao movimento da agulha em cada tecido foi outro comportamento considerado. O resultado dos experimentos mostrou que as opções de configurações presentes nas ferramentas táteis programáveis são capazes de representar esses comportamentos.

\section{Conclusão}
\label{sec:conclusão}

Dentre as contribuições deste trabalho está a reprodução virtual das principais sensações hápticas necessárias para simular a anestesia raquidiana. Foram feitos experimentos para avaliar a eficiência da simulação a partir do uso das ferramentas hápticas disponíveis. As configurações usadas foram estabelecidas com o intuito de representar as principais sensações experimentadas por anestesistas durante o procedimento da raquianestesia. Através de questionários objetivos procurou-se reduzir a subjetividade das respostas nas análises destas sensações e não induzir o avaliador com opções preestabelecidas. O sistema foi avaliado quanto a sua usabilidade de forma quantitativa através do método \textit{\acrshort{SUS}} \cite{Brooke2013} tendo como resposta a classificação de usabilidade boa. Especialistas fizeram uso do ambiente de simulação e responderam um questionário de avaliação. Estes informaram considerar como útil este tipo de treinamento. Os anestesistas recomendaram algumas melhorias visando aproximar o simulador ainda mais do procedimento real e desta forma poder ser de fato utilizado como ferramenta de treinamento de anestesia raquidiana. Na comparação com o simulador de \textcite{Farber2008} as avaliações do simulador desta tese obtiveram notas inferiores porém, como comentado na Seção~\ref{sec:avaliacaoEspecialistas}, não foi possível reproduzir o mesmo cenário e nem o mesmo equipamento (dispositivo háptico) nos testes. É importante ressaltar que, apesar das notas menores, o simulador desta tese tem duas das características principais de simuladores de punção mais completos que o simulador de \textcite{Farber2008}. Possibilitamos a palpação da coluna para escolha do ponto de inserção da agulha, esta característica só existe em outros dois dos simuladores computacionais estudados nesta tese (nenhum destes simula a anestesia raquidiana). A outra característica é que com o modelo 3D adaptável desenvolvido por nós é possível a simulação de múltiplas pacientes. O simulador de \textcite{Farber2008} só possui três (3) representações de pacientes.

Outra contribuição que é uma etapa fundamental para tornar a simulação mais real e adaptável a vários cenários de testes foi a construção de um modelo 3D do tronco de uma gestante de forma facilmente modificável via programação. A variabilidade de pacientes que pode ser configurada usando o modelo proposto é bastante vasta. Para uma maior reprodução do caso real no que diz respeito à espessura das principais camadas o modelo foi alimentado por um equação genérica desenvolvida neste trabalho. Foi também descrito o passo a passo das formas de uso da \textit{engine} da \textit{Unity} e do \textit{plugin} para importação do modelo e interação deste com o háptico na simulação. 

Em relação à área médica no treinamento de anestesia raquidiana a ideia principal é o uso de alguns níveis de treinamento com o simulador visando evitar que pacientes sejam submetidos a procedimentos deste tipo sem que o anestesista tenha alguma experiência prática prévia. Neste sentido a construção de um ambiente para treinamento de raquianestesia que apresenta \textit{feedbacks} durante e após cada procedimento bem como uma proposta de avaliação de desempenho por meio de notas foi trazido como contribuição. 

\section{Trabalhos Futuros}
\label{sec:trabFuturo}

Uma próxima etapa idealmente envolveria profissionais instrutores de procedimentos de raquianestesia de forma a validar a usabilidade do simulador o que permitiria a inclusão de novas funcionalidades ou remodelagem de funcionalidades existentes adequando este às necessidades de treinamento. Alguns exemplos de adequações a serem feitas mesmo sem esse contato com especialistas poderiam envolver por exemplo a representação do escorrimento do líquor quando a agulha fica por alguns segundos no espaço subaracnóideo e a modelagem de toda a paciente incluindo as demais partes do corpo para um ambiente ainda mais completo em termos de imersão. Ainda visando uma representação mais fiel da realidade a inclusão de \textit{feedback} de dor da paciente para possíveis movimentos bruscos de objetos perfurantes no seu corpo pode ser incluída, uma opção é usar um sintetizador de voz para este fim. A inclusão de óculos 3D traz benefícios em relação à imersão ainda que tendo o custo deste equipamento. É uma realidade possível em hospitais e universidades com melhor estrutura para treinamento.
Em relação à generalização e aproveitamento do trabalho feito na construção do modelo 3D é possível fazer o uso deste modelo para outros tipos de procedimentos que forem efetuados na mesma área do corpo (dorso). 

A aplicação de técnicas para reconhecimento de engajamento por parte do usuário \cite{Mitsis2022} é uma opção que poderia ser usada para alterar rumos do treinamento.
Outro caminho de trabalho futuro está no uso de realidade aumentada no lugar do dispositivo háptico usando a identificação da mão e incluindo esta no ambiente virtual. Nesta abordagem no momento da punção seria incluído a agulha ou seringa na mão da pessoa usando o simulador. Para tornar possível esta prática é preciso o estudo de formas de ``enganar'' o cérebro humano para que este perceba a resistência ao avançar da mão no ambiente virtual mesmo sem que algo físico impeça este movimento (como acontece no caso real e com a simulação com o háptico). Uma vez encontrada esta solução a abordagem ganharia em flexibilidade e teria o custo reduzido por conta da ausência da necessidade de existência do dispositivo háptico. Um rumo que envolve tecnologias recentes para esta área seria o de modelar uma nova arquitetura do ambiente de treinamento com a integração de conceitos de internet das coisas \cite{Ahmad2022}.