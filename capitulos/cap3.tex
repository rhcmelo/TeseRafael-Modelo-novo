\chapter{Trabalhos Relacionados} \label{cap:cap3}

%Os trabalhos apresentados neste capítulo vão desde simuladores que usam só \texit{phantoms} a simuladores computacionais. É interessante destacar que alguns simuladores computacionais são híbridos no sentido de que usam \textit{phantoms} como parte do processo de simulação. 

Os trabalhos apresentados neste capítulo vão desde abordagens de simulação de procedimentos de anestesia usando \textit{phantons} passando por simuladores computacionais bem como os diferentes tipos de modelagens de partes do corpo humano usados para simulações médicas.  

\textcite{Vaughan2013} citam trinta e um simuladores (entre computacionais e com uso de \texit{phantoms}). Destes, dezesseis são apenas epidurais, nove que permitem tanto a punção epidural quanto a raquidiana e seis apenas raquidianos. São discutidas as limitações e vantagens de cada um, de forma a identificar características desejáveis para serem incluídas em um simulador \cite{Vaughan2013}. No trabalho de \textcite{Isaacs2015} os autores citam uma pesquisa onde também são elencadas diversas características interessantes para um simulador epidural \cite{Isaacs2015}. Dentre os itens com mais relevância citados por \textcite{Isaacs2015} estão: coluna fisicamente palpável, representação da técnica de perda de resistência (do inglês inglês Lost of Resistence, \acrshort{LOR}) de forma realista (para solução salina e ar), ajuste da posição do paciente, mapeamento de características do paciente (obesidade, gravidez), \texit{feedback} a respeito da correta execução.
A principal vantagem dos modelos com uso de \texit{phantoms} é a presença física do manequim que representa o paciente. Uma das principais desvantagens é a dificuldade na variabilidade de cenários (pacientes) distintos. Outra desvantagem é a dificuldade na representação da diferença existente entre a resistência dos tecidos biológicos e os tecidos nos quais os modelos físicos são constituídos (usualmente borracha e plástico).
Os dispositivos hápticos têm melhorado muito ao longo dos últimos anos. Os modelos computacionais com hápticos têm como pontos a favor a versatilidade. As possibilidades computacionais da tecnologia háptica e as visualizações 3D de tempo real são outros pontos positivos associados a esta tecnologia. 
Soluções computacionais podem criar um modelo de forças de inserção da agulha. Para este fim pode-se usar como base as medições das forças aplicadas para inserção de agulhas em animais considerados próximos dos humanos, assim como em cadáveres e até mesmo voluntários reais \cite{Hiemenz1998, Holton2001,Langton1990,McKay2010,Naemura2009,Tran2009,Vaughan2012}. Estes modelos podem contemplar uma grande variabilidade de cenários através de ajustes de parâmetros. 

A seguir serão apresentados os simuladores de punções mais completos. Primeiro os que somente fazem uso de \textit{phantoms}, em seguida os que usam de ferramentas computacionais.  

\section{Simuladores que só usam \texit{phantoms}} \label{sec:SimuladoresPhantoms}

Os simuladores baseados em \texit{phantoms} (ou manequins) mais completos são listados aqui com suas principais características. As características positivas devem idealmente ser contempladas ou até melhoradas em novos simuladores. Todas as soluções listadas estão disponíveis atualmente. Elas permitem ao menos a simulação da anestesia peridural, possuem a coluna fisicamente palpável e permitem a escolha do ponto e ângulo de inserção da agulha. Todos esses trabalhos permitem que o procedimento seja feito com o paciente sentado ou deitado e o escoamento do fluido cérebro espinhal é simulado ao perfurar a dura-máter. As abordagens mediana e paramediana de inserção de agulha são possíveis em todos estes simuladores.

A solução \texit{SimULab Lumbar Epidural Trainer} possui até 3 variações de pacientes (normal, idoso e obeso). A Figura \ref{fig:simuladorSimulab} ilustra o uso deste simulador onde na esquerda da imagem (a) está a simulação da inserção da agulha no manequim e na imagem da direita (b) o uso do ultrassom. O material deste manequim foi produzido de forma a permitir o uso de ultrassom \cite{SimulabCorporation2008}. 

\begin{figure}[ht!]
    \centering
        \begin{tabular}{cc}
        \includegraphics[width=0.3\linewidth]{capitulos/figuras/simulab-insercao-agulha.jpg} & 
        \includegraphics[width=0.3\linewidth]{capitulos/figuras/simulab-ultrassom.jpg} 
        \\
        (a) & (b)
        \end{tabular}
    \caption{Demonstração de uso do simulador SimULab Lumbar Epidural Trainer \cite{SimulabCorporation2008}.}
    \label{fig:simuladorSimulab}
\end{figure}

O \textit{Blue Phantom Lumbar Puncture and Spinal Epidural} (Figura \ref{fig:bluePhantom}) também foi produzido em material que possibilita o uso do ultrassom. Na imagem (a) aparece o manequim e na (b) e (c) aparecem respectivamente o uso do ultrassom e o escorrimento do líquor. Este simulador possui somente duas opções de variação de pacientes (normal e obeso) \cite{BluePhantom2011}. 

\begin{figure}[ht!]
    \centering
        \begin{tabular}{ccc}
        \includegraphics[width=0.17\linewidth]{capitulos/figuras/BluePhatom-manequim.png} & 
        \includegraphics[width=0.3\linewidth]{capitulos/figuras/BluePhatom-ultrassom.jpg} 
        &
        \includegraphics[width=0.3\linewidth]{capitulos/figuras/BluePhatom-escorrimentoLiquor.jpg} 
        \\
        (a) & (b) & (c)
        \end{tabular}
    \caption{Blue Phantom Lumbar Puncture and Spinal Epidural \cite{BluePhantom2011}.}
    \label{fig:bluePhantom}
\end{figure}

O M43B \textit{Lumbar puncture simulator}-II (Figura \ref{fig:m43bSimulator}) permite 3 variações de pacientes (normal, idoso e obeso) e simula também a anestesia raquidiana \cite{KyotokagakuCo.2011}. Uma limitação deste é somente apresentar as vértebras L2 até L5 mas como vimos anteriormente, na seção \ref{sec:anestesiaRaquidiana}, estas são as vértebras mais comuns para anestesia raquidiana. A imagem apresenta o kit completo à esquerda, a palpação da coluna no centro e a inserção da agulha à direita.

\begin{figure}[ht!]
    \centering
        \begin{tabular}{ccc}
        \includegraphics[width=0.3\linewidth]{capitulos/figuras/m43b-kit.jpg} & 
        \includegraphics[width=0.3\linewidth]{capitulos/figuras/m43b-usoPalpacao.jpg} 
        &
        \includegraphics[width=0.3\linewidth]{capitulos/figuras/m43b-usoAgulha.jpg} 
        \\
        (a) & (b) & (c)
        \end{tabular}
    \caption{Kit e exemplo de uso do simulador M43B \texit{Lumbar puncture simulator}-II \cite{KyotokagakuCo.2011}.}
    \label{fig:m43bSimulator}
\end{figure}

Outro simulador que também permite a anestesia raquidiana é o \texit{Nasco Life/form® Spinal Injection Sim}. As vértebras L1 e L2 ficam visíveis externas ao corpo simulado pelo manequim. Ele somente permite a punção entre as vértebras L3 até L5 \cite{Nasco2008}. Na imagem da Figura \ref{fig:nascoSimulator} aparece um exemplo de uso deste simulador.

\begin{figure}[ht!]
    \centering
    \includegraphics[width=0.3\linewidth]{capitulos/figuras/nascoSimulator.png} 
    \caption{Demonstração de uso do simulador \textit{Nasco Life/form® Spinal Injection Sim} \cite{Nasco2008}.}
    \label{fig:nascoSimulator}
\end{figure}

A Tabela \ref{tab:comparacaoSimuladoresPhantoms} resume uma série de características importantes destes simuladores. 

\begin{table}[!ht]
\begin{center}
\caption{Comparação dos simuladores epidurais baseados em phantoms.}
\label{tab:comparacaoSimuladoresPhantoms}
%\begin{tabular}{|p{0.27\linewidth}|p{0.55\linewidth}|p{0.105\linewidth}|}
\begin{tabular}{|c|c|c|c|c|c|c|c|c|}
\hline
  & 
  \rotatebox{90}{Ano de desenvolvimento} & 
  \rotatebox{90}{Epidural (E) Raquianestesia (R)} & 
  \rotatebox{90}{Testado em hospital} & 
  \rotatebox{90}{Permite o uso de ultrassom} & 
  \rotatebox{90}{Variabilidade de pacientes} & 
  \rotatebox{90}{Coluna palpável} & 
  \rotatebox{90}{Escolha do ponto de inserção da agulha }  & 
  \rotatebox{90}{Abordagens mediana e paramediana} \\
\hline\hline
 SimULab Lumbar Epidural Trainer & 2008 & E &  & OK & 3 & OK & OK & OK \\
 {\begin{tabular}[c]{@{}c@{}}Blue Phantom Lumbar Puncture\\ and Spinal Epidural\end{tabular}} & 2011 & ER &  & OK & 2 & OK & OK & OK \\
 M43B Lumbar puncture simulator-II & 2010 & ER & OK &  & 3 & OK & OK & OK \\
 Nasco Life/form® Spinal Injection Sim & 2008 & ER &  &  & 3 & OK & OK & OK \\
\hline
\end{tabular}
\end{center}
\end{table}

\section {Simuladores computacionais}

Esta seção comenta os simuladores computacionais que possuem as características mais relevantes ao desenvolvimento pretendido.

O primeiro simulador epidural computacional, Epidural Sim \cite{Stredney1996} apresentou uma série de características interessantes. A representação do corpo do paciente em 3D teve a variabilidade de seus elementos obtida através de exames de \acrfull{RM}. Permite o uso de uma agulha real ligada a um dispositivo háptico. O modelo de forças foi baseado em medidas feitas durante inserções epidurais em porcos e cães combinados com opiniões de especialistas. Este simulador permite ainda a escolha do ponto de inserção da agulha assim como o seu ângulo. É disponibilizada uma interface de voz, o que possibilitava ser iniciado por comandos de voz do usuário assim como receber \texit{feedbacks} gerados pelo computador. A palpação da coluna não está disponível neste simulador. Chegou a ser testado por anestesistas que o consideraram muito mecânico, porém com potencial de melhora a partir de ajustes. Uma imagem do uso deste simulador pode ser vista na Figura \ref{fig:epiduralSim}. 

\begin{figure}[ht!]
    \centering
    \includegraphics[width=0.3\linewidth]{capitulos/figuras/epiduralSimulator.png} 
    \caption{Epidural Sim em uso \cite{Stredney1996}.}
    \label{fig:epiduralSim}
\end{figure}

O único simulador computacional estudado que possibilita a palpação da coluna é o \texit{Epidural Injection Simulator, EIS} \cite{Wilson2003}. Esta característica é atendida através de um equipamento físico que fica conectado a uma unidade de controle. O simulador possui uma interface gráfica que mostra, em tempo real, a progressão da agulha em cada camada de tecido conforme ela é inserida. Existem 6 variações de pacientes e um \texit{feedback} de forças configurável. Não permite a escolha do ponto de inserção da agulha nem o seu ângulo de inserção, que são fixos. Uma nova versão deste simulador foi desenvolvida com o nome de \texit{Epidural Injection Simulator Profile Manager}. Essa, além das funcionalidades do seu antecessor possibilita a criação de cenários customizados de pacientes além das 6 opções pré-configuradas. Estas customizações podem inclusive ser salvas para uso posterior. O \texit{feedback} em tempo real nesta nova versão pode ser visualizada no monitor do computador no lugar da unidade de controle \cite{CPRSavers&FirstAidSupply2018}. A Figura \ref{fig:EpiduralInjectionSimulator} exibe o visual do \texit{Epidural Injection Simulator Profile Manager} e do seu antecessor EIS que foi descontinuado com o lançamento da nova versão.

\begin{figure}[ht!]
    \centering
        \begin{tabular}{cc}
        \includegraphics[width=0.4\linewidth]{capitulos/figuras/epiduralInjectionSimulatorPM.jpg} & 
        \includegraphics[width=0.3\linewidth]{capitulos/figuras/epiduralInjectionSimulator.jpg} 
        \\
        (a) & (b)
        \end{tabular}
    \caption{Imagens de exemplo das duas versões do \texit{Epidural Injection Simulator}: (a) \texit{Versão mais nova: Profile Manager} (b) Versão antiga: EIS (descontinuado)  \cite{CPRSavers&FirstAidSupply2018}.}
    \label{fig:EpiduralInjectionSimulator}
\end{figure}

Em 2006 foi lançado o \textit{Mediseus® epidural simulator} (MedicVision Pty Ltd, Melbourne, Austrália). Este simulador apesar de ter sido descontinuado apresentava algumas características interessantes como a exibição completa do corpo permitindo rotações e zoom \cite{Mayooran2006}. A pele pode ser tornada transparente tornando visíveis as cinco vértebras modeladas. Este simulador usa um \textit{Phantom Omni} dentro de uma caixa patenteada \cite{Brien2007} para o \textit{feedback} das forças na agulha incluindo a medida da pressão de ar na seringa. A agulha é movida na tela em tempo real assim que o dispositivo é movimentado. O dispositivo pode ser conectado em qualquer laptop. Em uma avaliação feita sobre este simulador de 2007 \cite{Elks2007} ele não teve um bom retorno por parte dos anestesistas. O item com pior avaliação foi a sensação de LOR que somente 54\% dos especialistas julgaram como realista. Em um estudo posterior \cite{Lee2012} essa técnica foi avaliada com nota 4,7 de um valor máximo de 5 o que supõe uma representação próxima da sensação real esperada. Uma imagem deste simulador pode ser vista na Figura \ref{fig:mediseusSimulator}.

\begin{figure}[ht!]
    \centering
    \includegraphics[width=0.6\linewidth]{capitulos/figuras/mediseusSimulator.png} 
    \caption{Aparelho e visual do \textit{Mediseus® epidural simulator}  \cite{Mayooran2006}.}
    \label{fig:mediseusSimulator}
\end{figure}

O \textit{Spinal Anaesthesia Simulator} \cite{Albert2007,Dreifaldt2006}, faz o uso de um dispositivo háptico em conjunto com óculos 3D (Figura \ref{fig:spinalAnestesiaSim}). Permite tanto a simulação da anestesia peridural como a raquidiana. O modelo das costas é feito a partir de uma combinação de imagens de \acrfull{TC} e de \acrshort{RM}. Existem vários níveis de dificuldade configuráveis além de ser possível alterar o nível de visibilidade da pele. Faz uma análise do conhecimento proporcionando um \textit{feedback} ao usuário do nível de aprendizado e as habilidades adquiridas nos vários níveis de dificuldade disponíveis.

\begin{figure}[ht!]
    \centering
    \includegraphics[width=0.3\linewidth]{capitulos/figuras/spinalAnestesiaSim.png} 
    \caption{Forma de uso do \textit{Spinal Anaesthesia Simulator}  \cite{Dreifaldt2006}.}
    \label{fig:spinalAnestesiaSim}
\end{figure}

O EpiSim \cite{YantricInc2011} foi desenvolvido em 2008. Ele faz uso do háptico \textit{Phantom Premium} 1.0 além de um manequim e agulha epidural com uma seringa para recriar a sensação de perda de resistência. A espessura de cada camada de tecido é configurável a partir da interface podendo estas configurações serem salvas para uso futuro. É dado um \textit{feedback} visual a partir de cores, a cor verde indicando o tecido atual e a vermelha indica possíveis toques no osso. Um som de aviso é ouvido no caso de perfuração da dura-máter. É possível efetuar a gravação da inserção de uma agulha inclusive com as forças empregadas para posterior exibição. Esta etapa permite que um procedimento feito por especialistas seja visualizado em detalhes por profissionais inexperientes. A Figura \ref{fig:epiSim} ilustra este simulador.

\begin{figure}[ht!]
    \centering
    \includegraphics[width=0.4\linewidth]{capitulos/figuras/epiSim.png} 
    \caption{Imagem do simulador \textit{EpiSim} \cite{Frazzetto2011}.}
    \label{fig:epiSim}
\end{figure}

Um simulador de punção lombar criado por \textcite{Farber2008}  possui além do dispositivo háptico \textit{Phantom Premium} 1.0, gráficos anatômicos e tela estereográfica. Os movimentos de rotação e transversais são restritos quando a agulha está dentro do corpo. As forças hápticas são calculadas a partir de dados segmentados e tomografias \cite{Farber2008, Farber2009}. Uma abordagem de processamento de volume háptico \cite{Lundin2005} foi adaptada pelos autores para mapear as tomografias em forças, este método usa os vetores gradientes das imagens para este fim. O paciente virtual 3D é construído a partir de dados de tomografia de pacientes reais e de dados do projeto \textit{Visible Human}. As visualizações incluem uma anatomia 3D e 3 visualizações 2D mostrando os cortes ortogonais (transversal, frontal e sagital). É possível usar uma visão sob a perspectiva da agulha virtual. A Figura \ref{fig:farberSimVisual} ilustra as opções de visualizações citadas, a visão da agulha aparece na parte superior esquerda. A visão pode ser rotacionada e amplificada por meio de zoom. Possui uma boa impressão de profundidade no corpo virtual através da visão estéreo. Possibilita a variação entre 3 opções de pacientes virtuais. Pelas diversas tecnologias envolvidas este simulador tem um custo consideravelmente maior que os demais. Na Figura \ref{fig:farberSimDispositivos} é possível visualizar os dispositivos envolvidos no uso do simulador. Testes num estudo piloto demonstraram que os médicos treinados neste simulador se saíram melhor do que os que não tiveram acesso a ele \cite{Farber2009}. Para estes testes foi utilizada uma população de n=42 dividida em dois grupos de 21.

\begin{figure}[ht!]
    \centering
    \includegraphics[width=0.8\linewidth]{capitulos/figuras/farberSimVisual.png} 
    \caption{Interface do simulador \cite{Farber2009} com opções de visualização 2D e 3D disponíveis.}
    \label{fig:farberSimVisual}
\end{figure}

\begin{figure}[ht!]
    \centering
    \includegraphics[width=0.5\linewidth]{capitulos/figuras/farberSimDispositivos.png} 
    \caption{Uso do simulador \cite{Farber2009} demonstrando os dispositivos utilizados.}
    \label{fig:farberSimDispositivos}
\end{figure}

O \textit{Epidural Haptic Game Simulator}, EHGS \cite{Brazil2017}, fez uso de dispositivo háptico e elementos de jogos para trazer mais motivação para os usuários. Dentre os elementos de jogos estão presentes a divisão de etapas e atribuição de pontuação pelo correto cumprimento destas no procedimento de punção epidural. A simulação da técnica de LOR foi feita através da exibição de valores na tela ao clicar sobre um botão do háptico simulando o pressionamento do êmbolo da seringa. É possível escolher o ponto e ângulo de inserção da agulha. Não existe a opção de palpação da coluna para descoberta do local correto para efetuação da punção. Foi implementado um modelo de cálculo da espessura dos tecidos conforme peso, altura e idade do paciente de acordo com estudos em parturientes. A modelagem de forças de rigidez, atrito e corte necessárias para inserção e progresso da agulha nos tecidos foi desenvolvida com base em estudos destas forças para inserção de agulhas em tecidos de porcos e humanos. A ferramenta também disponibiliza uma forma manual de alteração destes parâmetros permitindo assim uma customização destas forças segundo as sensações do especialista. A Figura \ref{fig:brasilSimulator} mostra a interface do simulador EHGS em detalhes com o retorno de pontos obtidos, o tecido sendo perfurado no momento, a profundidade da ponta da agulha, as forças medidas e ângulo da agulha.

\begin{figure}[ht!]
    \centering
    \includegraphics[width=0.8\linewidth]{capitulos/figuras/brasilSimulator.png} 
    \caption{Interface do simulador EHGS \cite{Brazil2017}.}
    \label{fig:brasilSimulator}
\end{figure}

Um resumo das características importantes nos simuladores epidurais computacionais pode ser vista na Tabela \ref{tab:comparacaoSimuladoresComputacionais}.

\begin{sidewaystable}[!ht]
\begin{center}
\caption{Comparação dos simuladores computacionais.}
\label{tab:comparacaoSimuladoresComputacionais}
%\begin{tabular}{|p{0.27\linewidth}|p{0.55\linewidth}|p{0.105\linewidth}|}
\begin{tabular}{|c|c|c|c|c|c|c|c|c|c|c|c|}
\hline
  & 
  \rotatebox{90}{Ano de desenvolvimento} & 
  \rotatebox{90}{Realidade virtual} & 
  \rotatebox{90}{Epidural (E) Raquianestesia (R)} & 
  \rotatebox{90}{Feedback de forças} & 
  \rotatebox{90}{Testado em hospital} & 
  \rotatebox{90}{Baseado em dados medidos} & 
  \rotatebox{90}{Variabilidade de pacientes} & 
  \rotatebox{90}{Feedback para o usuário} & 
  \rotatebox{90}{Coluna palpável} & 
  \rotatebox{90}{Escolha do ponto de inserção da agulha }  & 
  \rotatebox{90}{Abordagens mediana e paramediana} \\
\hline\hline
 Epidural Sim & 1995 & 3D & E & Sim & OK & {\begin{tabular}[c]{@{}c@{}}Porcos e\\cães\end{tabular}} & Múltiplas &  &  & OK & OK \\
 {\begin{tabular}[c]{@{}c@{}}Epidural Injection \\Simulator Profile \\Manager\end{tabular}} & {\begin{tabular}[c]{@{}c@{}}2003\\(1a\\versão)\end{tabular}} &  & E & Customizado &  &  & {\begin{tabular}[c]{@{}c@{}}Múltiplas + salvar \\para uso posterior\end{tabular}} & {\begin{tabular}[c]{@{}c@{}}Habilidade \\rápida; Gráficos\end{tabular}} & OK & &  \\
 {\begin{tabular}[c]{@{}c@{}}Mediseus® epidural \\simulator\end{tabular}} & 2006 & 3D & E & Customizado & OK &  & 2 &  &  & &  \\
 {\begin{tabular}[c]{@{}c@{}}Spinal Anaesthesia \\simulator\end{tabular}} & 2006 & 3D & ER & Phantom &  & TC + RM &  & {\begin{tabular}[c]{@{}c@{}}Conhecimento\\dos níveis de\\dificuldade\end{tabular}} &  & OK & OK  \\
 EpiSim & 2008 & 3D & E & Phantom &  &  & {\begin{tabular}[c]{@{}c@{}}Múltiplas + salvar\\para uso posterior\end{tabular}} & Gravações &  & & OK \\
 {\begin{tabular}[c]{@{}c@{}}Simulador de punção  \\lombar de Färber\end{tabular}} & 2009 & 3D & ER & Phantom & OK & TC & 3 & Gravações &  & OK & OK \\
 EHGS & 2017 & 3D & E & Phantom &  & {\begin{tabular}[c]{@{}c@{}}Porcos e\\humanos\end{tabular}} & Múltiplas & Pontuações &  & OK &  \\
\hline
\end{tabular}
\end{center}
\end{sidewaystable}

\subsection {Comparação}

Enquanto os simuladores baseados em \textit{phantoms} mais generalistas possibilitam 3 variações de pacientes alguns dos simuladores computacionais mais completos possibilitam infinitas representações de pacientes através de ajustes de parâmetros \cite{Stredney1996, Wilson2003, YantricInc2011, Brazil2017}. No simulador que propomos aqui também possibilitamos infinitas representações como nestes simuladores. 

Em contrapartida apenas um dos simuladores computacionais desenvolvidos tratou a apalpação física da coluna \cite{Wilson2003}. Esta é uma característica desejável e importante que está presente em todos os simuladores baseados em \textit{phantoms}. A proposta desta tese se aproxima da abordagem do simulador \texit{Epidural Injection Simulator, EIS} \cite{Wilson2003} no que diz respeito a esta característica, uma vez que apresenta através de interação com o dispositivo háptico a possibilidade de apalpação da coluna porém sem a necessidade do item físico. Tudo é feito através de \acrshort{RV}. Desta forma a limitação de escolha do ponto de inserção da agulha e sua angulação presente em \texit{Epidural Injection Simulator, EIS} não existe no simulador aqui proposto e esta é outra característica muito importante presente na maioria dos modelos baseados em \textit{phantoms}. Este fator já estava incorporado nos principais simuladores computacionais.

Dois pontos positivos importantes nos simuladores computacionais são: a representação do \textit{feedback} de forças e a variabilidade dos tecidos dos pacientes ser feita baseada em dados reais de pacientes ou em modelos matemáticos extraídos de exames de pacientes. Para estes dois aspectos iremos utilizar o mesmo modelo matemático baseado em dados de porcos e humanos como em \tectcite{Brazil2017} porém, ao invés de fazer o uso de simplificações cilíndricas do formato do corpo e suas camadas internas usados por ele, geramos um modelo 3D de cada camada entre a pele e o último tecido alcançado representando de forma fiel a forma de cada camada. Para isto nos baseamos num corpo 3D interativo dos mais completos e mais precisos do ponto de vista científico \cite{BioDigitalInc2019}. Desta forma chegamos a um resultado visual similar ao da visão 3D do simulador de punção lombar de Färber sem a necessidade de imagens médicas como as usadas em \tectcite{Farber2009} e \textcite{Dreifaldt2006}. Ter uma representação correta das camadas internas do corpo na área de aplicação das punções é uma característica indispensável para que sejam corretamente adotadas as abordagens mediana e paramediana de inserção da agulha. O nosso simulador portanto permite que estas duas abordagens sejam adotadas de forma que agulha atravesse as camadas corretas em cada caso. 

A possibilidade de se fazer infinitas simulações, característica inerente a todos os simuladores computacionais, também é uma grande vantagem destes. Em simuladores baseados em \textit{phantoms}, ao menos algumas peças têm as suas vidas úteis comprometidas com o grande número de procedimentos que precisam ser efetuados. São necessárias muitas repetições de procedimentos para que o médico adquira as habilidades das técnicas de anestesia de neuroeixo \cite{Konrad1998}. Pelo estudo de \textcite{Kopacz1996} são necessárias em torno de 45 procedimentos de raquidiana e 60 de peridural para que o iniciante chegue a 90\% de sucesso no procedimento.

O \textit{feedback} é outro ponto muito importante que somente pode ser implementado por simuladores computacionais. Dentre os simuladores estudados várias formas foram utilizadas para essas respostas como: gravações de execuções para estudo posterior \cite{Farber2009, Frazzetto2011}, a descrição do conhecimento obtido assim como a graduação da qualidade da prática \cite{Wilson2003, Dreifaldt2006, Brazil2017}. Estas características são fundamentais para uma ferramenta de aprendizado tanto para o aprendiz quanto para o avaliador. O aprendiz se beneficia ao acompanhar o seu aprendizado e pode aumentar sua dedicação em caso de identificação de falhas em algum ponto específico da prática. O avaliador pode usar o retorno da ferramenta para quantificar o nível de habilidade do aprendiz e definir limites antes da prática em pacientes reais. ================== No simulador aqui proposto propomos diferentes níveis de dificuldade de procedimentos (no mínimo três) que precisam ser feitas com determinado nível de habilidade para que assim seja recebida uma nota de zero a dez relativo ao grau de assertividade do aprendiz na prática. A quantidade de procedimentos necessário antes de receber uma nota acima de seis (média) aumenta caso o aprendiz apresente muitos erros nos procedimentos e diminui com os procedimentos sendo feitos da forma correta o que de certa forma tem similaridades com os simuladores de \textcite{Wilson2003}, \textcite{Dreifaldt2006} e \textcite{Brazil2017}. 

=========================================
