
%%%% CAPÍTULO 1 - INTRODUÇÃO
%%
%% Deve apresentar uma visão global da pesquisa, 
%% incluindo: breve histórico, importância e
%% justificativa da escolha do tema, delimitações
%% do assunto, formulação de hipóteses e objetivos
%% da pesquisa e estrutura do trabalho.

% Perguntas que podem guiar a introdução - não necessariamente irá ter a resposta para tudo, isso depende da área.
% 1 - Qual é o contexto em que seu trabalho está inserido?
% 2 - Qual é o problema que motiva a existência deste trabalho?
% 3 - Qual é a visão geral da literatura sobre o problema e como é tratado
% 4 - Por que a solução na literatura não é o suficiente para ?
% 5 - Como seu trabalho trata o problema ?
% 6 - como seu trabalho foi avaliado para comprovar que tratou adequadamente o problema?
% 7 - De forma geral quais foram os resultados ?
% 8 - Quais foram as contribuições do seu trabalho?
% 9 -  Como o restante da Dissertação ou Tese está organizada ?


\chapter{Introdução}
\label{cap:introducao}

Nas anestesias raquidianas os anestesistas dependem da sua sensação tátil durante a inserção da agulha no paciente para a correta identificação do local de aplicação do líquido anestésico. O local de aplicação da raquidiana é conhecido como espaço subaracnóideo \cite{Miller2009}). Para que o anestesista reconheça a chegada da agulha neste local ele precisa reconhecer os tecidos perfurados no caminho dela. As anestesias possuem técnicas específicas para identificação dos seus espaços de aplicação. Para que os médicos dominem a técnica da anestesia raquidiana é estimado que são necessários 44 ± 6 repetições de execução deste tipo de procedimento \cite{Kopacz1996}. A confirmação de que o local adequado foi atingido na anestesia raquidiana é feita através da observação do vazamento, por meio da agulha de punção, do liquido cérebro espinhal ou cefalorraquidiano (\textit{líquor}). As Figuras~\ref{fig:puncaoLombar} e~\ref{fig:gotejamentoLiquor} ilustram dois momentos importantes da anestesia raquidiana retirados do vídeo de \textcite{Londero2018} disponível no link\footnote{\url{https://www.youtube.com/watch?v=Dl8ijvHVTuY&ab_channel=CLAUDINEILONDERO}}. Na Figura~\ref{fig:puncaoLombar} é mostrado o momento de inserção da agulha para punção lombar e na Figura~\ref{fig:gotejamentoLiquor} é mostrado o vazamento, através da agulha de punção, do \textit{líquor}, o que acontece alguns segundos após a agulha estar corretamente posicionada no espaço subaracnóideo. Neste tipo de anestesia é usada uma agulha de menor diâmetro do que a agulha utilizada na anestesia epidural \cite{Miller2009}. O ultrassom é uma ferramenta eficiente para auxílio na determinação do espaço onde a agulha precisa ser inserida \cite{Helayel2010, Soni2019} bem como na definição da espessura das diversas camadas de tecido \cite{Klingensmith2022}. Existem inclusive soluções desenvolvidas para interpretação de imagens de ultrassom que vem sendo estudadas para substituir a apalpação do anestesista na determinação do ponto de inserção da agulha \cite{Ni2021}. Porém, o uso de equipamentos de ultrassom para este fim não é uma realidade em muitos centros no Brasil \cite{Hamaji2016}. O uso deste equipamento ou qualquer outra técnica \cite{Berde2022}, portanto não faz parte do treinamento de muitas faculdades de medicina para anestesias raquidianas. A determinação do ponto de inserção da agulha no treinamento, assim como no procedimento real em pacientes, é comumente feita fazendo uso de referências anatômicas através da apalpação da crista ilíaca do paciente. A crista ilíaca pode ser observada nas imagens da Figura~\ref{fig:cristaIliaca} \cite{Moura2019}. 

\begin{figure}[!ht]
   \centering
   \includegraphics[width=0.6\linewidth]{capitulos/figuras/2.PuncaoLombar.png}
   \caption{Punção lombar com agulha de raquianestesia  \cite{Londero2018}.}
   \label{fig:puncaoLombar}
\end{figure}

\begin{figure}[!ht]
   \centering
   \includegraphics[width=0.6\linewidth]{capitulos/figuras/3.GotejamentoLiquor.png}
   \caption{Gotejamento do \textit{líquor}, indicação do local correto para a raquianestesia \cite{Londero2018}.}
   \label{fig:gotejamentoLiquor}
\end{figure}

\begin{figure}[ht!]
    \centering
        \begin{tabular}{cc}
        \includegraphics[width=0.55\linewidth]{capitulos/figuras/crista-iliaca-pelve-ossos-ligamentos.png} & 
        \includegraphics[width=0.35\linewidth]{capitulos/figuras/crista-iliaca-ossos-quadril.png} 
        \\
        (a) & (b)
        \end{tabular}
    \caption{Ilustração da crista ilíaca através de duas imagens dos ossos da pelve \cite{Moura2019}. Nas imagens a indicação de elementos que não estão diretamente relacionados à crista ilíaca foram embaçados de forma a simplificar a visualização desta: (a) Vista frontal (b) Vista lateral.}
    \label{fig:cristaIliaca}
\end{figure}

A principal abordagem de treinamento para técnicas de anestesia envolve a observação da aplicação das técnicas por anestesistas experientes \cite{Vrillon2022}. Estes orientam verbalmente os aprendizes conforme cada um dos passos é executado. Adicionalmente a isto são usados: desenhos 2D, cadáveres para demonstração do procedimento, apresentação de vídeos das técnicas sendo executadas em pacientes, visualização 3D \cite{Vrillon2022} e técnicas de simulação. No que diz respeito ao treinamento das sensações táteis além do uso de cadáveres alguns simuladores fazem uso de bonecos com tecidos artificiais, denominados \textit{phantoms}, que simulam pacientes \cite{Dreifaldt2006, KyotoKagaku2015, Mashari2018}. Um ponto negativo importante no uso de \textit{phantoms} e de cadáveres, talvez o principal, é a baixa representatividade no que diz respeito a reprodução da situação real, pois estes oferecem uma baixa variabilidade de cenários (variações possíveis dos corpos dos pacientes) para treinamento. Esta fato fica notório na análise destes tipos de simuladores feita na Seção \ref{sec:SimuladoresPhantoms}. A maior quantidade de tipos de corpos representado pelos \textit{phantoms} nos simuladores estudados foi de 4 enquanto é sabido que a variabilidade das estruturas corporais mesmo em uma população pequena é muito maior do que isso. Outro aspecto relevante no uso de \textit{phantoms} é a necessidade de reposição de peças que se desgastam com o uso e podem ter custos altos. Estes são alguns dos motivos para que em diversos hospitais a primeira experiência do anestesista em treinamento seja efetuada diretamente em um paciente \cite{Aggarwal2009, Grantcharov2008, Smith2005, Watterson2007}. Esta prática, apesar de ser efetuada sob supervisão direta de médicos experientes, pode trazer riscos para as pessoas que são anestesiadas (pacientes) e uma maior propensão a inseguranças por parte dos aprendizes \cite{Elmofty2017}. 

O uso de simuladores para adquirir certo grau de habilidade antes de iniciar o procedimento em pacientes minimiza os riscos tanto para o aprendiz quanto para o paciente não só na anestesia \cite{Escobar-Castillejos2016, Yunoki2018} como em diversas outras áreas da medicina \cite{Akhtar2014, Alvarez-Lopez2020, Hamm2022}. A existência de diversos cenários em simuladores como os que usam \acrfull{RV} auxilia e motiva o ensino e possibilita ao aprendiz ter experiência com situações mais variadas assim como aumenta a segurança dos alunos. Essa técnica tem também a vantagem de possibilitar que a avaliação do desempenho destes seja feita de forma automatizada e padronizada \cite{Willis2014}. 

Estes simuladores com frequência usam diferentes níveis variando as dificuldades \cite{Ullrich2012}. Possibilitam a redução ou eliminação de custos de manutenção de equipamentos e laboratórios físicos bem como evitam a necessidade de estruturação de laboratórios \cite{Silva2018}. 
Esta variabilidade de cenários dificilmente aconteceria na vida real em centros onde o ensino é feito diretamente em pacientes \cite{Udani2015}. No cenário de treinamento diretamente em pacientes a experiência inicial de cada anestesista pode ser muito distinta uma vez que estas dependem da estrutura corporal do paciente atendida por cada residente. Esta não é uma abordagem ideal para treinamentos uma vez que cria uma dependência no que diz respeito à experiência tátil dos anestesistas novatos em uma variável que não está sob o controle do profissional responsável pelo treinamento. Cada aluno pode vir a ter uma gama diferente de experiências a depender das características físicas dos pacientes que este teve suas primeiras experiências. Isto impacta o nivelamento do ensino.

Diversos simuladores utilizam dispositivos de força háptica (\textit{force feedback}) para auxiliar o aprendiz a experimentar fisicamente as sensações de resistência modeladas para os tecidos ao praticar procedimentos médicos. Este tipo de abordagem é usada em procedimentos médicos de um modo geral \cite{Escobar-Castillejos2016, Patel2021} assim como no caso mais específico dos procedimentos de anestesia \cite{Vaughan2013, Collaco2021}. Existem muitas outras formas de como o uso de ferramentas computacionais pode auxiliar no campo da anestesia. Um exemplo é no controle automatizado de quanto anestésico aplicar a partir de respostas de medições dos níveis de consciência do paciente \cite{Mendez2009}. Outro exemplo é uso da imersão \acrshort{RV} durante a cirurgia em conjunto com a anestesia como forma de reduzir a dor e estimular o relaxamento diminuindo a ansiedade e possivelmente a quantidade de anestésico necessário \cite{Eijlers2019}.

\textcite{Correa2019}, na sua análise do estado da arte, relataram que as avaliações da percepção humana são pouco exploradas no campo da interação háptica para treinamento de inserção de agulhas. Eles também citam a predominância de testes subjetivos para validação das soluções propostas por parte dos usuários. Alguns trabalhos fazem uso de análises subjetivas usando gráfico de profundidade da agulha versus tempo como em \textcite{Magill2010}. 

O trabalho desta tese foi iniciado com a apresentação do simulador de \textcite{Brazil2017thesis} para um anestesista com o intuito de agregar a este simulador outro tipo de anestesia, no caso a anestesia raquidiana. A proposta inicial desta tese propunha a utilização destes simuladores para a avaliação de ganho de conhecimento no treinamento de novos anestesistas com o uso do simuladores em comparação com a ausência do seu uso. A opinião do especialista foi a de que este tipo de avaliação seria mensurável de forma mais concreta para o simulador de raquianestesia. A partir de estudos do estado da arte foi observado que os simuladores que possibilitam a anestesia raquidiana não contemplam algumas das principais características desejáveis para a correta representação do procedimento como por exemplo a apalpação da coluna para determinação do ponto de inserção da agulha. Com isto um simulador de anestesia raquidiana foi construído visando atender as principais demandas do treinamento das sensações envolvidas neste procedimento. Na literatura estudada foram adotadas diversas formas de abordar o problema do treinamento de anestesias regionais. Simuladores deste tipo possuem muitas características relevantes o que faz com que o foco no atendimento em algumas geralmente venha a comprometer outras. Grande parte dos simuladores computacionais de anestesia estudados apresenta como opção para o usuário somente a simulação de anestesias epidurais e não de raquianestesias que é o foco desta tese.

A apresentação de cenários de treinamento virtuais que simulam a possibilidade de visualização e sentimentos táteis que são vivenciados no procedimento real visa aproximar a prática de treino virtual da posterior prática em pacientes reais. Desta forma, possibilita um maior sentimento de segurança por parte dos anestesistas aprendizes. A aplicação de transparência em camadas foi uma das técnicas que foi incluída no ambiente virtual de treinamento criado nesta tese. Esta funcionalidade permite a visão do interior do corpo o que facilita o entendimento do aprendiz no que diz respeito à teoria do procedimento. Esta conexão da teoria com a prática é um grande trunfo no uso da \acrfull{RV} em simuladores para treinamento. Neste caso, o realismo na apresentação dos elementos envolvidos no treinamento tem um papel importante. Seja a representação 3D da agulha, do corpo ou das camadas internas que possam ser visualizadas. 

\section{Definição do problema}
\newtheorem{prob}{Problema}

Esta tese de doutorado define a seguinte questão de pesquisa a ser estudada e solucionada por este trabalho.

\begin{prob}
\label{prob:simuladorCasoReal}
    É possível criar uma ferramenta virtual para treinamento médico que auxilie no treinamento do procedimento de raquianestesia a partir do uso de realidade virtual e com dispositivo háptico de forma a simular este procedimento desde a palpação da coluna (para determinação do ponto de inserção da agulha)?
\end{prob}

Dentre os simuladores de raquianestesia computacionais atuais não existe um que contemple na simulação a palpação do corpo pelo médico anestesista para determinação do ponto de inserção da agulha. Esta opção existe para somente dois simuladores de anestesia epidurais estudados nesta tese sendo que somente em um deles este procedimento é feito de maneira 100\% virtual como propomos neste trabalho. O simulador epidural que possibilita simulação de apalpação não teve a sua solução avaliada por especialistas. 

\section{Objetivos}
\label{sec:objetivos}

Esta tese tem o seguinte objetivo geral:

Propor e desenvolver um ambiente de treinamento virtual no que se refere às técnicas de anestesia raquidiana em gestantes desde a apalpação para determinação do ponto de inserção da agulha até a administração do anestésico no local correto. A ideia é prover um aprendizado prático, didático e mais completo dos anestesistas sem incorrer em risco para os pacientes e estresse para os anestesistas em treinamento. Este treinamento será padronizado no sentido das técnicas que precisarão ser dominadas pelos usuários treinados. O sistema irá fazer com que os usuários que demonstrarem melhor desenvoltura nas etapas iniciais evoluam mais rapidamente pelas etapas de demonstração de habilidades já entendidas e aprendidas. 

Dentre os objetivos específicos pode-se destacar:
\begin{enumerate}
\item Possibilitar variações das situações possíveis de ocorrer em relação às características físicas das pacientes; 
\item Desenvolver modelagens que mapeiam as características físicas na visualização; 
\item Possibilitar visualização dos tecidos internos no momento da anestesia para auxílio no aprendizado inicial. 
\item Utilizar técnicas de realidade virtual na representação da paciente e dos equipamentos usados no procedimento em ambiente 3D interativo;
\item Empregar dispositivos hápticos como meio de interação para simular os sentimentos táteis do médico (de forma semelhante ao procedimento real em treinamento);
\item Dar retorno ao usuário em relação as ações feitas corretas e incorretas no que diz respeito ao procedimento visando auxiliar na evolução do seu desempenho.
\end{enumerate}

\section{Contribuições da Tese}
\label{sec:contribuicoes}

Uma das principais contribuições deste trabalho envolve a reprodução virtual das principais sensações hápticas necessárias para simular a anestesia raquidiana. Outra contribuição foi a construção de um modelo 3D real da parte lombar do corpo de uma gestante (tecidos entre a pele e a cauda equina) de forma facilmente modificável via programação. A variação dos tamanhos das principais camadas do modelo foi alimentada por um equação genérica criada e detalhada neste trabalho. Foi criado então um ambiente para treinamento de raquianestesia que apresenta \textit{feedbacks} durante e após cada
procedimento bem como uma proposta de avaliação de desempenho por meio de notas.

\section{Estrutura da Tese}
\label{sec:estrutura}

O restante do texto está estruturado da seguinte forma. O Capítulo~\ref{cap:cap2} comenta os principais conceitos e tecnologias envolvidas no desenvolvimento do ambiente de treinamento proposto.

O Capítulo~\ref{cap:cap3} contém os trabalhos relacionados a esta tese assim como o posicionamento deste trabalho frente aos demais.

No Capítulo~\ref{cap:cap4} é apresentada a especificação do ambiente de treinamento que foi desenvolvido durante esta tese. 

A implementação do ambiente de treinamento que foi desenvolvido durante esta tese está descrita no Capítulo~\ref{cap:cap5}. 

O Capítulo~\ref{cap:cap6} apresenta os experimentos que foram feitos e uma avaliação destes em relação aos seus resultados.

Por fim, o Capítulo~\ref{cap:cap7} conclui o trabalho, apresentando as conclusões, realçando as contribuições desta tese e apontando as limitações e os  trabalhos futuros.


