
%%%% CAPÍTULO 1 - INTRODUÇÃO
%%
%% Deve apresentar uma visão global da pesquisa, 
%% incluindo: breve histórico, importância e
%% justificativa da escolha do tema, delimitações
%% do assunto, formulação de hipóteses e objetivos
%% da pesquisa e estrutura do trabalho.

% Perguntas que podem guiar a introdução - não necessariamente irá ter a resposta para tudo, isso depende da área.
% 1 - Qual é o contexto em que seu trabalho está inserido?
% 2 - Qual é o problema que motiva a existência deste trabalho?
% 3 - Qual é a visão geral da literatura sobre o problema e como é tratado
% 4 - Por que a solução na literatura não é o suficiente para ?
% 5 - Como seu trabalho trata o problema ?
% 6 - como seu trabalho foi avaliado para comprovar que tratou adequadamente o problema?
% 7 - De forma geral quais foram os resultados ?
% 8 - Quais foram as contribuições do seu trabalho?
% 9 -  Como o restante da Dissertação ou Tese está organizada ?


\chapter{Introdução}
\label{cap:introducao}

Nas anestesias raquidianas os anestesistas dependem do seu sentimento tátil durante a inserção da agulha no paciente para a correta identificação do local de aplicação do líquido anestésico. O local de aplicação da raquidiana é conhecidos como espaço subaracnóideo \cite{Miller2009}). Para que o anestesista reconheça a chegada da agulha neste local ele precisa reconhecer os tecidos ultrapassados por ela. As anestesias possuem técnicas específicas para identificação dos seus espaços de aplicação. Para que os médicos dominem a técnica da anestesia raquidiana é estimado que são necessários 44 ± 6 repetições de execução deste tipo de procedimento. A confirmação de que o local adequado foi atingido na anestesia raquidiana é feita através da observação do vazamento, através da agulha de punção, do liquido cérebro espinhal ou cefalorraquidiano (\textit{líquor}). As Figuras~\ref{fig:puncaoLombar} e ~\ref{fig:gotejamentoLiquor}  ilustram dois momentos importantes da anestesia raquidiana retirados de um video. Na Figura~\ref{fig:puncaoLombar} é mostrado o momento de inserção da agulha para punção lombar e na Figura~\ref{fig:gotejamentoLiquor} é mostrado o vazamento, através da agulha de punção, do (\textit{líquor}), o que acontece alguns segundos após a agulha estar corretamente posicionada no espaço subaracnóideo. Neste tipo de anestesia é usada uma agulha de menor diâmetro do que a agulha utilizada na anestesia epidural \cite{Miller2009}. O ultrassom é uma ferramenta eficiente para auxilio na determinação do espaço onde a agulha precisa ser inserida \cite{Helayel2010}, mas o uso deste tipo de equipamento não é uma realidade em muitos centros do Brasil \cite{Hamaji2016}. O uso deste equipamento, portanto não faz parte do treinamento de muitas faculdades de medicina para anestesias raquidianas. Este treinamento é feito através da palpação da crista ilíaca do paciente.



\section{Opções de Citação}
\label{sec:citacoes}

Fonte dos dados sobre citação direta e indireta do site \url{https://www.tecmundo.com.br/tutorial/834-aprenda-a-usar-as-normas-da-abnt-citacao-2-de-4-.htm}, autora \textcite{xavier2020}

São comandos de citação deste template usando o pacote abntex2cite:

\textbf{textcite\{referencia\}} retorna por exemplo: \textcite{kitchenham2009systematic}

\textbf{textcite[n de pag.]\{referencia\}}  retorna por exemplo uma citação com a pagina de referencia: \textcite[p.20]{kitchenham2009systematic}

\textbf{cite\{referencia\}} retorna por exemplo: \cite{kitchenham2009systematic}

\textbf{cite[n de pag.]\{referencia\}}  retorna por exemplo uma citação com a pagina de referencia: \cite[p.20]{kitchenham2009systematic}

Obs: para as aspas ficarem corretas usar no inicio 2 crase `` e no final 2 apóstrofos ''

\subsection{Citação direta}

Referente as citações diretas existem 2 formas, sendo que os textos devem estar sempre entre aspas (``  texto'') ``a citação direta é a transcrição textual fiel de parte de um conteúdo de uma obra''  \cite{xavier2020}. Como os exemplos neste paragrafo onde ``a chamada pelo nome do autor, quando feita no final da citação, deve apresentar-se entre parênteses, contendo o sobrenome do autor em letra maiúscula, seguido pelo ano de publicação e página em que o texto se encontra'' \cite{xavier2020}. 

Também existe a citação direta quando o o autor está no inicio da citação segundo \textcite{xavier2020} ``Assim, o sobrenome do autor deve ser digitado normalmente, com a primeira letra em maiúscula e as demais em minúsculo, seguido do ano e página em que o texto se encontra, sendo estas informações apesentadas entre parênteses''.


Se a citação direta for com mais de 03 linhas há uma configuração específica.

Conforme descreve \textcite{xavier2020}:

\begin{quoting}[rightmargin=0cm,leftmargin=4cm] % o comando leftmargin tem o recuo de 4cm
\begin{singlespace} %espaço simples 
{
\footnotesize %comando para a fonte ficar tamanho 10
As citações com mais de três linhas devem ter um tipo de destaque diferente: é necessário reduzir o tamanho da fonte, podendo ser para 10 ou 11 e também é preciso aplicar um recuo de 4cm em relação à margem esquerda. Ao final, a citação com mais de três linhas terá a seguinte apresentação — observe que ela não tem aspas.
}
\end{singlespace} %final do comando de espaço simples
\end{quoting} % fim do comando quoting


\subsection{Citação Indireta}

Escreve com um monte de citações diretas deixa o texto muito carregado de aspas (``  ''). Nesta situação pode-se chegar a conclusão que não se deseja escrever o texto com as palavras exatas do autor, tornando assim o texto mais fluido. Para isso pode usar o texto de lido como base e escrever com suas palavras \cite{xavier2020}.


Tantos as citações diretas quanto as indiretas podem conter mais de um autor, como por exemplo, \cite{xavier2020, da2005}. Essa forma de citação também pode estar no inicio da conforme \textcite{xavier2020, da2005} as citações pode conter mais de uma referencia no inicio da frase. 



 \subsection{O comando \textit{apud}}

% ***** O TEXTO DESTA SUBSEÇÃO É DOS AUTORES ABAIXO, PARA DESCREVER O COMANDO apud 
% % % % % % % % % % % % % % % % % % % % % % % % % % % % % % % % %
%%%% Classe LaTex - MDT-UFSM-2015 - ver. 1.08 (06/09/2017)   %%%%
%%%% adaptado da versao de Fabio Natanael Kepler (sem data)  %%%%
%%%% Grupo de desenvolvimento:                               %%%%
%%%% Franciano Scremin Puhales, Josue Sehnem,                %%%%
%%%% Pablo Eli Soares de Oliveira e Diogo Machado Custodio   %%%%
%%%% Contato: latexufsm@googlegroups.com                     %%%%
% % % % % % % % % % % % % % % % % % % % % % % % % % % % % % % % % 
 
  \par A citação \textit{apud} ocorre quando você cita algum autor através de outra obra, sem ter consultado-a propriamente. Neste caso a citação é feita da seguinte forma:
   
   A frase original vem de ABC, mas você leu em YYY, não achou o texto original de ABC, então faz um \textit{apud}, neste comando primeiro vem antes vem o autor original (ABC) e depois quem citou (YYY)
  
  \begin{center}
  \rule{0.5\textwidth}{1pt}\\ 
  $\backslash apud\{material\_citado\_no\_material\_lido\}\{material\_lido\}$ \\
  \end{center}
\begin{verbatim}
Sobre a circulação geral da atmosfera pode-se dizer que os ventos do norte
não movem moinhos a frase original vem de da2005 mas quem falou foi xavier2020, antes vem o autor original e depois quem citou \apud{da2005}{xavier2020}. 
\end{verbatim}
  
  Sobre a circulação geral da atmosfera pode-se dizer que os ventos do nortenão movem moinhos \apud{xavier2020}{da2005}.
  
  Com citação da pagina de origem \apud[p.5]{xavier2020}{da2005}.
  
\begin{center}\rule{0.5\textwidth}{1pt}\end{center}  
  \par Nesse caso, na bibliografia só constará a obra consultada e não aquele referenciada pela obra. Para que isso ocorra naturalmente, a obra consultada deve ser incluída normalmente no arquivo referencias.bib enquanto a obra referenciada indiretamente deve ser incluída com a opção \textit{@hidden}, conforme o modelo de referências\footnote{Isto é um teste de nota de rodapé}.

      \subsubsection{\textit{Apud on line}}

      
      \par O \textit{textapud} se aplica da mesma maneira que o \textit{apud} descrito anteriormente. O termo \textit{on line} é alusivo ao \textit{$\backslash$textcite$\{$label$\}$} definido no abntex. Nesse caso a citação é feita da seguinte forma:
      \begin{center}
      \rule{0.5\textwidth}{1pt}
            $\backslash textapud\{material\_citado\_no\_material\_lido\}\{material\_lido\}$ \\
	    \end{center}

 \begin{verbatim}
Segundo \textapud{xavier2020}{da2005}, os ventos do
norte não movem moinhos.
\end{verbatim}

            Segundo \textapud{xavier2020}{da2005}, os ventos do norte não movem moinhos.
            
            Com citação da pagina de origem  \textapud[p.20]{xavier2020}{da2005}, os ventos do norte não movem moinhos.



\section{Siglas e Abreviaturas}
\label{sec:siglas}

Para usar siglas tem o pacote \acrfull{ACR} que utiliza os comandos:

acrfull - \acrfull{ONU} 

acrshort - \acrshort{ONU}

acrlong -  \acrlong{ONU}



\section{Figuras}

As figuras podem estar centralizadas ou dependerá da forma de escrita do texto. 

Exemplo de Figura: Ver Figura~\ref{fig:exefig}.

\begin{figure}[!ht]
   \centering
   \includegraphics[width=0.3\linewidth]{capitulos/figuras/exefig.eps}
   \caption{Exemplo de figura}
   \label{fig:exefig}
\end{figure}



\section{Tabelas}

Pode-se utilizar um gerador de tabelas on-line, que agiliza o processo de escrita. Como exemplo temos o site \url{https://www.tablesgenerator.com/}. 


Exemplo de Tabela: ver Tabela~\ref{tab:exetab}.


\begin{table}[!ht]
\begin{center}
\caption{Exemplo de tabela}
\label{tab:exetab}
\begin{tabular}{|c |c |}
\hline
\textbf{\textbf{Dado 1}} & \textbf{Percentual}\\
\hline\hline
Tipo 1 & 0,6 \\
Tipo 2 & 0,8 \\
Tipo 3 & 1,0 \\
Tipo 4 & 0,3 \\
\hline
\end{tabular}
\end{center}
\end{table}
