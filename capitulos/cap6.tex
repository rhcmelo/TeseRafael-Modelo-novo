\chapter{Considerações Finais} \label{cap:cap6}

\section{Conclusão}
\label{sec:conclusão}

A contribuição principal deste trabalho está na reprodução virtual das principais sensações hápticas necessárias para simular a anestesia raquidiana. Outra contribuição que é uma etapa fundamental para tornar a simulação mais real e adaptável a vários cenários de testes foi a construção de um modelo 3D do tronco de uma gestante de forma facilmente modificável via programação. A variabilidade de pacientes que pode ser configurada usando o modelo proposto é bastante vasta. Foi também descrito o passo a passo das formas de uso da \textit{engine} da \textit{Unity} e do \textit{plugin} para importação do modelo e interação deste com o háptico na simulação. 

Foram feitos experimentos para avaliar a eficiência da simulação a partir do uso das ferramentas hápticas disponíveis com configurações sugeridas com o intuito de representar as principais sensações experienciadas por anestesistas durante o procedimento da raquianestesia. Através de questionários objetivos procuramos reduzir a subjetividade das respostas nas análises destas sensações.

Em relação a área médica no treinamento de anestesia raquidiana a ideia principal é o uso de alguns níveis de treinamento com o simulador visando evitar que pacientes sejam atendidos por pessoas sem antes participar de um treinamento prático.

\section{Trabalhos Futuros}
\label{sec:trabFuturo}

Outras partes do corpo, inclusão de \textit{feedback} de dor do paciente para movimentos bruscos  