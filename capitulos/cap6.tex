\chapter{Considerações Finais} \label{cap:cap6}

O uso de um simulador no treinamento pode ajudar a atenuar riscos de falhas da raquianestesia relacionados a habilidades médicas não adquiridas corretamente. Simulamos aqui os comportamentos necessários para representar a agulha espinhal penetrando através dos vários ligamentos e tecidos do corpo. Um simulador de raquianestesia de alta fidelidade requer recursos como uma coluna palpável, a capacidade de acomodar várias posições do paciente, inclinações ajustáveis de inserção da agulha e alguns outros aspectos principais. O objetivo aqui era simular corretamente as sensações táteis essenciais da punção da agulha durante o procedimento de raquianestesia. A identificação das transições entre os tecidos foi um dos comportamentos levado em consideração por ser imprescindível na representação do caso real e, portanto, no treinamento.
A variação da resistência ao movimento da agulha em cada tecido foi outro comportamento considerado. Os testes mostraram que as configurações presentes nas ferramentas táteis programáveis são capazes de representar esses comportamentos.


===== ver conclusão artigo EMBC =====


\section{Conclusão}
\label{sec:conclusão}

A contribuição principal deste trabalho está na reprodução virtual das principais sensações hápticas necessárias para simular a anestesia raquidiana. Outra contribuição que é uma etapa fundamental para tornar a simulação mais real e adaptável a vários cenários de testes foi a construção de um modelo 3D do tronco de uma gestante de forma facilmente modificável via programação. A variabilidade de pacientes que pode ser configurada usando o modelo proposto é bastante vasta e para uma maior reprodução do caso real no que diz respeito ao tamanho das principais camadas a variação dos seus tamanhos no modelo foi alimentada por um equação genérica criada e detalhada neste trabalho. Foi também descrito o passo a passo das formas de uso da \textit{engine} da \textit{Unity} e do \textit{plugin} para importação do modelo e interação deste com o háptico na simulação. 

Foram feitos experimentos para avaliar a eficiência da simulação a partir do uso das ferramentas hápticas disponíveis. Usamos configurações que foram estabelecidas com o intuito de representar as principais sensações experienciadas por anestesistas durante o procedimento da raquianestesia. Através de questionários objetivos utilizando perguntas de respostas abertas procurando reduzir a subjetividade das respostas nas análises destas sensações e não induzir o avaliador com opções preestabelecidas.  

Em relação a área médica no treinamento de anestesia raquidiana a ideia principal é o uso de alguns níveis de treinamento com o simulador visando evitar que pacientes sejam atendidos por pessoas sem antes participar de um treinamento prático.

Uso de realidade aumentada no lugar do dispositivo háptico usando a identificação da mão e incluindo esta no ambiente virtual. Nesta abordagem no momento da punção seria incluído a agulha ou seringa na mão da pessoa usando o simulador. Para tornar esta prática é preciso o estudo de formas de "enganar" o cérebro humano para que este perceba a resistência ao avançar da mão no ambiente virtual mesmo que sem que algo físico impeça este movimento (como acontece no caso real). Uma vez encontrada esta solução a abordagem ganharia em flexibilidade e teria o custo reduzido por conta da ausência da necessidade de existência do dispositivo háptico.

\section{Trabalhos Futuros}
\label{sec:trabFuturo}

Destacamos a seguir algumas dentre as diversas possibilidades de trabalhos de extensão do que propomos aqui. A modelagem de toda a paciente incluindo as demais partes do corpo para um ambiente ainda mais completo em termos de imersão. Ainda visando uma representação mais fiel da realidade a inclusão de \textit{feedback} de dor da paciente para possíveis movimentos bruscos de objetos perfurantes no seu corpo pode ser incluída, uma opção é usar um sintetizador de voz para este fim. A inclusão de óculos 3D traz benefícios em relação a imersão ainda que tendo o custo deste equipamento. É uma realidade possível em hospitais e universidades com melhor estrutura para treinamento.

Em relação a generalização e aproveitamento do trabalho feito na construção do modelo 3D é possível fazer o uso deste modelo para outros tipos de procedimentos que forem efetuados na mesma área do corpo (dorso). 